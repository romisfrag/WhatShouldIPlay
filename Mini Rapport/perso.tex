\section{Compte rendu personnel (Sylvain Ung)}

J'ai déjà eu avant ce projet une première expérience en développement mobile Android : les techniques de programmation sur cette plate-forme ne m'étaient donc pas totalement inconnues, notamment la notion d'activités au sein d'une application. Néanmoins, mes connaissances étaient plutôt superficielles, ce projet m'a donc permis de les approfondir. J'ai particulièrement apprécié la découverte et l'intégration de bibliothèques externes pour répondre à des besoins spécifiques qui ne pouvaient pas être satisfaits par la SDK Android de base, comme la gestion des GIF.
\paragraph{}
Mes principales contributions au développement de l'application concernait surtout la partie interface utilisateur par la production d'éléments graphiques pertinents et la disposition des layouts qui composent les activités. En effet, l'objectif principal est de rendre l'application agréable à utiliser et intuitive.

\newpage
\section{Compte rendu personnel (Roman Delgado)}

De part mon parcours, cette unité d'enseignement m'a confronté pour la première fois à la conception d'un projet comportant une interface graphique. En effet, je n'avais jamais développé d'applications Android. Cela a donc été l'occasion pour moi d'apprendre le fonctionnement du MVC\footnote{Modèle-Vue-Contrôleur}. Pour cette raison, l'expérience de mon camarade Sylvain a été d'une aide importante et nous a permis de gagner du temps.
J'ai d'abord été en charge de réaliser l'architecture globale du programme et de déterminer la façon dont nous allions récupérer et stocker la liste des cartes qui constitue les données élémentaires manipulées par notre application.

Pour répondre aux problématiques les plus importantes liées au développement de l'application nous nous sommes toujours concertés et avons travaillé en constante coopération.
Nous nous somme donc répartis le travail au fur et à mesure : Sylvain plus centré sur les problématiques liées à l'interface graphique et pour moi plus sur le traitement et le stockage des données.

Personnellement ce projet m'a énormément appris et à permis de consolider ma manière d'utiliser de nouveaux environements de développement. De plus, cela m'a permis de pouvoir commencer un projet personnel d'application Android en parallèle en plus de celle-ci.